\documentclass{article}
\usepackage[utf8]{inputenc}


\begin{document}

We implement the reverse Monte-Carlo method described in details in \citep{yong2016}.
The idea is to propagate a certain number of unpolarized light rays backward from the instrument and see where they end up.
The radiant flux of the ground at the end point is associated to the ray.
Once the ray touches the ground, we can trace it from the ground to the instrument to compute its polarisation at each scattering event.
Then we can just add all rays together to get the polarisation due to multiple scattering.\\

This has the huge advantage over forward monte carlo simulation to reduce th number of useless rays drastically, as all computed rays reach the detector.\\

An unpolarised ray of undefined intensity is propagated from the instrument detector along its line of sight.
At every predifined step of length $l$, the rays has a given probability to be scattered.
This probability is defined at every altitude as the scattering cross-section of the air and aerosols.
From \cite{bucholtz1995}, we have the Rayleigh scaterring cross-section per volume of atmosphere:
\begin{equation}
\beta(\lambda, z) = \beta_0(\lambda) \frac{P(z)}{P_0} \frac{T_0}{T(z)}\,,
  \label{eq:beta_z}
\end{equation}
with $P(z)$ and $T(z)$ the atmospheric pressure and temperature profiles.
$P_0 = 101\,325$ Pa and $T_0 = 288.15$ K are the pressure and temperature at sea level, and
\begin{linenomath*}\begin{equation}
    \beta_0(\lambda) = A_{ray}\lambda^{-(B+C\lambda + D/\lambda)}\,,
    \label{eqn:sqlaw}
\end{equation}\end{linenomath*}
with $\lambda$ the wavelength (in $\mu$m).

For now, I take as the probability of scattering for the ray:
\begin{equation}
  p(sca) = \beta l
\end{equation}



\end{document}
